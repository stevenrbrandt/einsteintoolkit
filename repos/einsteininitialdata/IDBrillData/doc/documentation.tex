\documentclass{article}

% Use the Cactus ThornGuide style file
% (Automatically used from Cactus distribution, if you have a 
%  thorn without the Cactus Flesh download this from the Cactus
%  homepage at www.cactuscode.org)
\usepackage{../../../../doc/latex/cactus}

\begin{document}

\title{IDBrillData}
\author{Carsten Gundlach, Gabrielle Allen}
\date{$ $Date$ $}

\maketitle

% Do not delete next line
% START CACTUS THORNGUIDE

\begin{abstract}
This thorn creates time symmetric initial data for Brill
wave spacetimes.  It can create both axisymmetric data (in a 3D
cartesian grid), as well as data with an angular dependency.
\end{abstract}

\section{Purpose}

The purpose of this thorn is to create (time symmetric) initial data
for a Brill wave spacetime.  It does so by starting from a
three--metric of the form originally considered by Brill
\begin{equation}
ds^2 = \Psi^4 \left[ e^{2q} \left( d\rho^2 + dz^2 \right)
+ \rho^2 d\phi^2 \right] =\Psi^4 \hat{ds}^{2},
\label{eqn:brillmetric}
\end{equation}
where $q$ is a free function subject to certain regularity and
fall-off conditions, $\rho=\sqrt{x^2+y^2}$ and $\Psi$ is a conformal
factor to be solved for.

Thorn {\tt IDBrillData} provides three choices for the $q$ function: 
an exponential form, ({\tt IDBrillData::q\_function = "exp"})
\begin{equation}
q = a \; \frac{\rho^{2+b}}{r^2} e^{-\left( \frac{z}{\sigma_z} \right)^2}
e^{-(\rho - \rho_0)^2} \left[ 1 + d \frac{\rho^m}{1 + e \rho^m}
\cos^2 \left( n \phi + \phi_0 \right) \right]
\end{equation}
a generalized form of the $q$ function first written down by Eppley
({\tt IDBrillData::q\_function = "eppley"}) 
\begin{equation}
q = a \left( \frac{\rho}{\sigma_\rho} \right)^b \frac{1}{1 + \left[
\left( r^2 - r_0^2 \right) / \sigma_r^2 \right]^{c/2}}\left[ 1 + d \frac{\rho^m}{1 + e \rho^m}
\cos^2 \left( n \phi + \phi_0 \right) \right]
\end{equation}
and the (default) Gundlach $q$ function which includes the Holz form
({\tt IDBrillData::q\_function = "gundlach"})
\begin{equation} 
q = a \left( \frac{\rho}{\sigma_\rho} \right)^b e^{-\left[
\left( r^2 - r_0^2 \right) / \sigma_r^2 \right]^{c/2}} \left[ 1 + d \frac{\rho^m}{1 + e \rho^m}
\cos^2 \left( n \phi + \phi_0 \right) \right]
\end{equation}

Substituting the metric into the Hamiltonian constraint gives an
elliptic equation for the conformal factor $\Psi$ which is then
numerically solved for a given function $q$:
\begin{equation}
\hat{\nabla} \Psi - \frac{\Psi}{8} \hat{R} = 0
\end{equation}
where the conformal Ricci scalar is found to be
\begin{eqnarray}
\hat{R} = -2 \left(e^{-2q} (\partial^2_z q + \partial^2_\rho q) + 
\frac{1}{\rho^2} (3 (\partial_\phi q)^2 + 2 \partial_\phi q)\right)
\end{eqnarray}
Assuming the initial data to be time symmetric means that the momentum
constraints are trivially satisfied.

In the case of axisymmetry (that is $d=0$ in the above expressions for
$q$), the Hamiltonian constraint can be written as an elliptic
equation for $\Psi$ with just the flat space Laplacian,
\begin{equation}
\nabla_{flat} \Psi + \frac{\Psi}{4} (\partial_z^2 q + \partial_\rho^2 q) = 0
\end{equation}
If the initial data is chosen to be {\tt
ADMBase::initial\_data = "brilldata2D"} then this elliptic equation
is solved rather than the equation above.


\section{Generating Initial Data with IDBrillData}

Brill initial data is activated by choosing the {\tt CactusEinstein/ADMBase}
parameter {\tt initial\_data} to be {\tt brilldata}, or for the case of
axisymmetry {\tt brilldata2D} can also be used.

The parameter {\tt IDBrillData::q\_function} chooses the form of the
$q$ function to be used, defaulting to the Gundlach expression.

Additional {\tt IDBrillData} parameters for each form of $q$ fix the 
remaining freedom:

\begin{itemize}

\item Exponential $q$: {\tt IDBrillData::q\_function = "exp"}

$(a, b,\sigma_z,\rho_0)=$ ({\tt exp\_a, exp\_b, exp\_sigmaz,exp\_rho0})

\item Eppley $q$: {\tt IDBrillData::q\_function = "eppley"}

$(a, b,\sigma_\rho, r_0,\sigma_r,c)=$ ({\tt eppley\_a, eppley\_b, eppley\_sigmarho, eppley\_r0, eppley\_sigmar, eppley\_c})

\item Gundlach $q$: {\tt IDBrillData::q\_function = "gundlach"}

$(a, b,\sigma_\rho, r_0,\sigma_r,c)=$ ({\tt gundlach\_a, gundlach\_b, gundlach\_sigmarho, gundlach\_r0, gundlach\_sigmar, gundlach\_c})

\item Non-axisymmetric part for each choice of $q$

$(d, m, e, n, \phi0)=$ ({\tt brill3d\_d, brill3d\_m, brill3d\_e, brill3d\_n, brill3d\_phi0})

\end{itemize}

Note that the default $q$ expression is
$$
q = {\tt gundlach\_a} \quad \rho^2 e^{-r^2} 
$$

{\tt IDBrillData} can use the elliptic solvers (type LinMetric)
provided by {\tt CactusEinstein/EllSOR},\\ {\tt AEIThorns/BAM\_Elliptic},
or {\tt CactusElliptic/EllPETSc} to solve the equation resulting from 
the Hamiltonian constraint. 
In all cases the parameter {\tt thresh} sets the threshold for the elliptic
solve. The choice of elliptic solver is made 
through the parameter {\tt brill\_solver}:

\begin{itemize}
  
\item {\tt sor}: Understands the Robin boundary condition, additional 
parameters control the maximum number of iterations ({\tt sor\_maxit}).
  
\item {\tt bam}: {\tt BAM\_Elliptic} does not properly implement the
elliptic infrastructure of {\tt EllBase}, and the {\tt BAM\_Elliptic}
parameter to use the Robin boundary condition must be set independently
of \\{\tt IDBrillWave::brill\_bound}.

\end{itemize}

\section{Notes}

Thorn {\tt IDBrillData} understands both the ``{\tt physical}'' and ``{\tt
static conformal}'' {\tt metric\_type}. In the case of a conformal
metric being chosen, the conformal factor is set to $\Psi$. Currently
the derivatives of the conformal factor are not calculated, so that
only {\tt staticconformal::conformal\_storage = "factor"} is
supported.

\section{References}

\subsection{Specification of Brill Waves}

\begin{enumerate}

\item Dieter Brill, {\bf Ann. Phys.}, 7, 466, 1959.

\item Ken Eppley, {\bf Sources of Gravitational Radiation}, edited by L. Smarr (Cambridge University Press, 
Cambridge, England, 1979), p. 275.

\end{enumerate}

\subsection{Numerical Evolutions of Brill Waves}

\begin{enumerate}

\item {\it Gravitational Collapse of Gravitational Waves in 3D Numerical Relativity}, 
 Miguel Alcubierre, Gabrielle Allen, Bernd Bruegmann, Gerd Lanfermann, Edward Seidel, Wai-Mo Suen, Malcolm Tobias,
{\bf Phys. Rev. D61}, 041501, 2000.

\end{enumerate}

% Do not delete next line
% END CACTUS THORNGUIDE

\end{document}
