% Hisaaki Shinkai shinkai@atlas.riken.go.jp <- PL use this address
%-----------------------------------------------------------------------
% KTsol.tex version 0.1  19980603
% Hisaaki Shinkai shinkai@wurel.wustl.edu
%-----------------------------------------------------------------------
%00000000111111111122222222223333333333444444444455555555556666666666777
%-----------------------------------------------------------------------
\documentstyle[11pt]{article}
% A4 tate -----------------------------
\topmargin      0.3in
\oddsidemargin  0.0cm
\evensidemargin 0.0cm
\textwidth     16.0cm
\textheight    23.0cm
\headsep        0.0in
%--- US letter size
\topmargin     -0.0in
\headsep        0.0in
\oddsidemargin   0.0in
\evensidemargin  0.0in
\textwidth       6.5in
\textheight      8.5in
%-----------------------------------------------------------------------
%------------------------- macro for begin-eqs
\def\non{\nonumber \\}
\def\nonn{\nonumber \\ &&}
\def\be{\begin{equation}}
\def\en{\end{equation}}
\def\bear{\begin{eqnarray}}
\def\enar{\end{eqnarray}}
\def\beas{\begin{eqnarray*}}
\def\enas{\end{eqnarray*}}
%--------------->>>>>>>>>>>> commands for number of eqs. (1.1a)(1.1b)...
% following definition is for book or report style
%\renewcommand{\theequation}{\thechapter.\theenumi\alph{equation}}
\def\bera{ \setcounter{enumi}{\value{equation}}
            \addtocounter{enumi}{1}
            \setcounter{equation}{0}
            \renewcommand{\theequation}{\theenumi\alph{equation}}
            \begin{eqnarray} }
\def\enra{ \end{eqnarray}
            \setcounter{equation}{\value{enumi}}
            \renewcommand{\theequation}{\arabic{equation}}  }
%-----------------------------------------------------------------------
\def\non{\nonumber \\}
\def\nonn{\nonumber \\ &&}
\def\dsp{\displaystyle}
\def\mova{\left( M \over a\right)}
%------------------------- macro for lists
\def\been{\begin{enumerate}}
\def\enen{\end{enumerate}}
\def\beit{\begin{itemize}}
\def\enit{\end{itemize}}

\def\pl{ \partial}
\def\half{{1 \over 2}}
%-----------------------------------------------------------------------
\begin{document}
\noindent
filename="KTsol.tex" HShinkai ({\tt shinkai@atlas.riken.go.jp})

\begin{center}

{\Large\bf Multi Black Hole solutions}

\end{center}
\begin{flushright}
version 0.1 ~~~
19980603 Hisaaki Shinkai
\end{flushright}

\section{Majumdar-Papapetrou solution}
Majumdar-Papapetrou (MP)
solutions\cite{MP} is a multi-black-hole solution to Einstein's
equation.
Each black holes are charged maximally, $Q=M$,  and
the balance between gravitational attraction and
electrostatic repulsion among the black holes causes each to maintain
its position relative to the others eternally.
The MP solutions are given by
\be
ds^2=-{1 \over \Omega^2} dt^2+ \Omega^2(dx^2+dy^2+dz^2),
\label{MPmetric}
\en
$$\mbox{where}~~
\Omega=1+\sum_{i=1}^N {M_i \over  r_i},~~ \mbox{and}~~
  r_i=\sqrt{(x-x_i)^2+(y-y_i)^2+(z-z_i)^2}.
$$
where $M_i$ and $(x_i, y_i, z_i) \in {\bf R}^3$ are masses and
locations of black holes.


\section{Kastor-Traschen solution}


The  Kastor-Traschen (KT) solutions \cite{KT} is the cosmological
version of MP solution.  This is a multi-black-hole solution
  to Einstein's
equation with cosmological constant,  contains arbitrary many
$Q=M$ black holes that participate in an overall de Sitter expansion
or contraction.  In the $\Lambda \rightarrow 0$
limit, the KT solutions reduce to the MP solution.
 
To write the KT
metric, we first choose $(x_i, y_i, z_i) \in {\bf R}^3$,
$i=1,2,\cdots,N$ for locations of black holes, then
\be
ds^2=-{1 \over \Omega^2} dt^2+a(t)^2 \Omega^2(dx^2+dy^2+dz^2),
\label{KTmetric}
\en
$$\mbox{where}~~
\Omega=1+\sum_{i=1}^N {M_i \over a r_i},~~ a=e^{Ht}, ~~~
H=\pm \sqrt{\Lambda \over 3}.$$
$$\mbox{and}~~ r_i=\sqrt{(x-x_i)^2+(y-y_i)^2+(z-z_i)^2}$$
where we interpret $M_i$
as the mass of the $i{\rm th}$ black hole, although we have
neither an asymptotically flat region nor event horizons available
to convert this naive interpretation into a rigorous one.

If $H<0$, then the solution represents ``incoming" charged BHs.
If $H>0$, then the solution represents ``outgoing" charged WHs.


The principal null directions of KT solutions are illustrated in
\cite{KTpnd}.
The horizon structure of KT solutions are discussed in \cite{KThorizon}.


%-----------------------------------------------------------------------
\newpage
\appendix
\section{Reissner-Nordstr{\o}m-de Sitter solution }
It will be interesting to see such a cosmological extention for
a single black hole case.
The global structure of this solution is discussed in
\cite{RNdShorizon}.

\subsection{Reissner-Nordstr{\o}m-de Sitter solution (static coord) }
\be
ds^s=-V(R)dT^2 + {1 \over V(R)} dR^2 + R^2 d\Omega^2
\en
$$\mbox{where}~~
V(R)=1-{2M\over R}+{Q^2 \over R^2}-{\Lambda \over 3} R^2. $$
\subsection{Reissner-Nordstr{\o}m-de Sitter solution (cosmological coord) }
By the transformation
$$
a(t) r =  R-M, ~~~
t= T+h(R), ~~~
{dh \over dR} = - {HR^2 \over (R-M)V(R)}
$$
and setting $Q=M$, we will get the cosmological coordinate version of
RNdS as
\be
ds^2=-{1\over F^2}dt^2 + {a^2(t) F^2} ( dr^2 +  r^2 d\Omega^2)
\en
$$\mbox{where}~~
F=1+{M\over ar}, ~~~  a=e^{Ht}, ~~~
H=\pm \sqrt{\Lambda \over 3}.$$.

The horizons appeare at
$$
r_\pm = {1\over 2 a(t) |H| } ( 1 - 2 M |H| \pm \sqrt{1 - 4 M |H|} )
$$
which are corresponds to de Sitter horizon ($r_+$) and
outer BH horizon  ($r_-$),  respectivly.




%%**********************************************************************
%23456789012345678901234567890123456789012345678901234567890123456789012

\baselineskip .15in
\begin{thebibliography}{99}

\bibitem{MP}
S. D. Majumdar, Phys. Rev. {\bf 72}, 930 (1947); A. Papapetrou,
Proc. R. Ir. Acad. Sect. {\bf A51}, 191 (1947);
J. B. Hartle and S. W. Hawking, Commun. Math. Phys. {\bf 26}, 87
(1972).


\bibitem{KT}
D. Kastor and J. Traschen,  Phys. Rev. {\bf D47}, 5370 (1993).

\bibitem{KTpnd}
L. Gunnarsen, H. Shinkai and K. Maeda, Class. Quantum Grav. {\bf 12},
133 (1995).

\bibitem{KThorizon}
K. Nakao, T. Shiromizu and S. A. Hayward,
Phys. Rev. {\bf D52}, 796 (1995).

\bibitem{RNdShorizon}
D. R. Brill and S. A. Hayward
Class. Quant. Grav. {\bf 11}, 359 (1994).
 
\end{thebibliography}

\end{document}
