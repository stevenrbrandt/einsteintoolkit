\documentclass{article}
\usepackage{../../../../doc/latex/cactus}
\begin{document}

% The title of the document (not necessarily the name of the Thorn)
\title{Hydro\_InitExcision - Initial Excision for Hydro Evolutions}

% The author of the documentation - on one line, otherwise it does not work
\author{Andrea Nerozzi, Frank L\"offler}

\maketitle

% Use the Cactus ThornGuide style file
% (Automatically used from Cactus distribution, if you have a 
%  thorn without the Cactus Flesh download this from the Cactus
%  homepage at www.cactuscode.org)

% Do not delete next line
% START CACTUS THORNGUIDE

%\newcommand{\eqref}[1]{(\ref{#1})}

% Add an abstract for this thorn's documentation
\begin{abstract}
  This thorn initialises the excision-spacemask for GRHydro to
  specified forms. It is based on a similar version used to initialise
  the excision-mask for Whisky code.
\end{abstract}

% The following sections are suggestive only.
% Remove them or add your own.

\section{Using This Thorn}

Using this thorn is quite easy:
You have to activate it in your parameter file and specify at least the
following:

\begin{verbatim}
hydro_initexcision::hydro_initexcision = "yes"
\end{verbatim}

In addition to that you may want to specify the type of excision zone
you want to have. You can do this by

\begin{verbatim}
hydro_initexcision::hydro_initexcision_type = <TYPE>
\end{verbatim}

where \begin{verbatim}<TYPE>\end{verbatim} is one of the following list:
\begin{itemize}
 \item "x-axis"
 \item "y-axis"
 \item "z-axis"
 \item "box"
 \item "sphere"
\end{itemize}

\section{Thorn Source Code}

This was initially written by Andrea Nerozzi and later changed
by Frank L\"offler.

\section{Thorn Documentation}

This documentation was largely written by Frank L\"offler.

% Do not delete next line
% END CACTUS THORNGUIDE

\end{document}
