\documentclass{article}

% Use the Cactus ThornGuide style file
% (Automatically used from Cactus distribution, if you have a 
%  thorn without the Cactus Flesh download this from the Cactus
%  homepage at www.cactuscode.org)
\usepackage{../../../../doc/latex/cactus}

\begin{document}

\title{IOJpeg}
\author{Gabrielle Allen, Thomas Radke}
\date{$ $Date$ $}

\maketitle

% Do not delete next line
% START CACTUS THORNGUIDE

\begin{abstract}
Output method using the jpeg format
\end{abstract}

\section{Use}

Thorn IOJpeg provides 2D images from grid functions in jpeg format, these
images are currently intended to be used for two purposes:

\begin{itemize}

\item{Monitoring from Web Server} Jpeg images can be directly visualised
in the {\it viewport} provided by thorn {\tt CactusConnect/HTTPDExtra}. For 
the images to be advertised to the Web Server the parameter {\tt iojpeg::mode = ``remove''} must be used.

\item{Constructing Movies} A series of jpeg movies can easily
 be used to construct a time series movie (e.g. using the Unix utilities
{\tt convert} or {\tt xanim}). In this case the parameter {\tt iojpeg::mode = ``standard''} must be used.

\end{itemize}

Note that the parameter {\tt iojpeg::mode} determines whether a jpeg image is ccreated and kept for individual timesteps, or whether only the image from the current data is kept. Only in the second case is the jpeg file {\tt advertised} (otherwise there are potentially thousands of files advertised to e.g. the {\tt HTTPD} thorn). Also note that using the standard mode and creating jpegs every iteration can quite quickly lead to {\it inode} problems on a machine.

The steerable parameter {\tt IOJpeg::refinement\_factor} determines whether the
resulting JPEG images (of same size as the underlying grid) should be refined
by a certain factor. If refinement is enabled ({\tt IOJpeg::refinement\_factor} $>$ 1) an interpolation will be done to enlarge the images to the requested size. For this case, a thorn providing local interpolation operators must be activated (eg. thorn {\tt CactusBase/LocalInterp}).

We are planning to develop this thorn more to provide more features (eg 
a range of 2D images from a 3D dataset, add possibility to save images for a movie and advertise current image).

% Do not delete next line
% END CACTUS THORNGUIDE

\end{document}
