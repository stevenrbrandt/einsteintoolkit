% *======================================================================*
%  Cactus Thorn template for ThornGuide documentation
%  Author: Ian Kelley
%  Date: Sun Jun 02, 2002
%  $Header$
%
%  Thorn documentation in the latex file doc/documentation.tex
%  will be included in ThornGuides built with the Cactus make system.
%  The scripts employed by the make system automatically include
%  pages about variables, parameters and scheduling parsed from the
%  relevant thorn CCL files.
%
%  This template contains guidelines which help to assure that your
%  documentation will be correctly added to ThornGuides. More
%  information is available in the Cactus UsersGuide.
%
%  Guidelines:
%   - Do not change anything before the line
%       % START CACTUS THORNGUIDE",
%     except for filling in the title, author, date, etc. fields.
%        - Each of these fields should only be on ONE line.
%        - Author names should be separated with a \\ or a comma.
%   - You can define your own macros, but they must appear after
%     the START CACTUS THORNGUIDE line, and must not redefine standard
%     latex commands.
%   - To avoid name clashes with other thorns, 'labels', 'citations',
%     'references', and 'image' names should conform to the following
%     convention:
%       ARRANGEMENT_THORN_LABEL
%     For example, an image wave.eps in the arrangement CactusWave and
%     thorn WaveToyC should be renamed to CactusWave_WaveToyC_wave.eps
%   - Graphics should only be included using the graphicx package.
%     More specifically, with the "\includegraphics" command.  Do
%     not specify any graphic file extensions in your .tex file. This
%     will allow us to create a PDF version of the ThornGuide
%     via pdflatex.
%   - References should be included with the latex "\bibitem" command.
%   - Use \begin{abstract}...\end{abstract} instead of \abstract{...}
%   - Do not use \appendix, instead include any appendices you need as
%     standard sections.
%   - For the benefit of our Perl scripts, and for future extensions,
%     please use simple latex.
%
% *======================================================================*
%
% Example of including a graphic image:
%    \begin{figure}[ht]
%       \begin{center}
%          \includegraphics[width=6cm]{MyArrangement_MyThorn_MyFigure}
%       \end{center}
%       \caption{Illustration of this and that}
%       \label{MyArrangement_MyThorn_MyLabel}
%    \end{figure}
%
% Example of using a label:
%   \label{MyArrangement_MyThorn_MyLabel}
%
% Example of a citation:
%    \cite{MyArrangement_MyThorn_Author99}
%
% Example of including a reference
%   \bibitem{MyArrangement_MyThorn_Author99}
%   {J. Author, {\em The Title of the Book, Journal, or periodical}, 1 (1999),
%   1--16. {\tt http://www.nowhere.com/}}
%
% *======================================================================*

% If you are using CVS use this line to give version information
% $Header$

\documentclass{article}

% Use the Cactus ThornGuide style file
% (Automatically used from Cactus distribution, if you have a
%  thorn without the Cactus Flesh download this from the Cactus
%  homepage at www.cactuscode.org)
\usepackage{../../../../doc/latex/cactus}

\begin{document}

% The author of the documentation
\author{Ian Hawke \textless ih@maths.soton.ac.uk\textgreater}

% The title of the document (not necessarily the name of the Thorn)
\title{General equations of state}

% the date your document was last changed, if your document is in CVS,
% please use:
%    \date{$ $Date$ $}
\date{March 15 2005}

\maketitle

% Do not delete next line
% START CACTUS THORNGUIDE

% Add all definitions used in this documentation here
%   \def\mydef etc

% Add an abstract for this thorn's documentation
\begin{abstract}
  This thorn provides a generalized equation of state interface.
\end{abstract}

% The following sections are suggestive only.
% Remove them or add your own.

\section{Equations of state}

In solving the equations of hydrodynamics there are five independent
variables (typically $\rho, v^i, \epsilon$ for the rest mass density,
velocity and specific internal energy) and further variables, such as
the pressure $P$ which depend on the independent variables. Such
dependent variables are specified by an {\em equation of state} (EOS)
that closes the system of equations. Simple EOSs depend only on the
independent variables and are usually given in terms of the pressure,
i.e.\ $P = P(\rho, \epsilon)$ which then implicitly defines other
variables such as the temperature $T$, pressure derivatives $P_{\rho},
P_{\epsilon}$ and the speed of sound $c_s$.

However there is no physical reason to prefer such simple EOSs. A
realistic EOS will depend on the composition of the fluid as well as
the independent variables. This thorn is meant to provide an interface
so that a hydrodynamics code can depend on such a realistic EOS
without having to know about the details of the composition --
without, in fact, needing an knowledge of the additional variables
required by the EOS at all.

\section{Calling an EOS}

The most efficient way of using the EOS is to call it to set grid
functions. The interface is simple on the face of it; one function
call will give you a {\em call handle}, and another function call will
actually set the EOS. The complexity comes in setting up the key value
tables involved in the function calls.

\begin{FunctionDescription}{EOS\_SetupCall}
  \label{Whisky_Dev_EOSBase_General_EOS_SetupCall}

  Sets up an equation of state call. If the call handle is stored
  (which is recommended) then this only needs to be performed once at
  startup time, and the call handle can be reused.

  \begin{SynopsisSection}
    \begin{Synopsis}{C}
\begin{verbatim}
CCTK_INT call_handle = EOS_SetupCall(CCTK_INT table_handle)
\end{verbatim}
    \end{Synopsis}
    \begin{Synopsis}{Fortran}
\begin{verbatim}
CCTK_INT call_handle = EOS_SetupCall(CCTK_INT table_handle)
\end{verbatim}
    \end{Synopsis}
  \end{SynopsisSection}

  \begin{ResultSection}
    \begin{ResultNote}
      The {\tt call\_handle} return must be stored for later use in
      the EOS call. The value should be $\geq 0$. Typically a level
      0 warning will be given rather than returning an error code.
    \end{ResultNote}
    \begin{Result}{$\geq 0$}
      The call handle, to be stored for later use.
    \end{Result}
  \end{ResultSection}
  
  \begin{ParameterSection}
    \begin{Parameter}{table\_handle}
      The handle of an existing key value table. Appropriate
      arguments will be described elsewhere.
    \end{Parameter}
  \end{ParameterSection}
  
  \begin{Discussion}
    The complexity comes in setting up the key-value table to be
    passed through. Effectively any arguments can be given that will
    be interpreted by the EOS, but the following are expected (note
    capitalization - it is always useful to set up the table using the
    CASE INSENSITIVE flag):
    \begin{itemize}
    \item {\tt EOS Name}: This must correspond to the name that the
      EOS thorn has registered.
    \item {\tt Independent variable names}: This is a string (a
      CCTK\_STRING) of whitespace separated names describing the
      independent variables. Although there is no particular values
      for these ``names'' required by this base thorns the following
      conventions for variable names have been used in existing thorns
      (name in parantheses is the variable name in {\sc whisky}):
      \begin{itemize}
      \item {\tt Rho}: Rest mass density ({\tt rho})
      \item {\tt SpecificInternalEnergy}: Specific internal energy
        ({\tt eps})
      \item {\tt Pressure}: Pressure ({\tt press})
      \item {\tt DPressureDRho}: Partial derivative of pressure with
        respect to rest mass density, all else being fixed ({\tt
          dpdrho})
      \item {\tt DPressureDSpecificInternalEnergy}: Partial derivative
        of pressure with respect to specific internal energy, all else
        being fixed ({\tt dpdeps})
      \item {\tt c\_s\^{}2}: Speed of sound squared ({\tt cs2}).
      \end{itemize}
    \item {\tt Dependent variable names}: A string describing the
      dependent variables, following the same syntax as the {\tt
        Independent variable names}
    \item {\tt N independent variables}: The number of independent
      variables
    \item {\tt N dependent variables}: The number of dependent
      variables
    \item {\tt Independent GFs}: An array of integers containing the
      indices of the grid functions corresponding to the independent
      variables. E.g., if the first dependent variable name is {\tt
        Rho} which is contained in the grid function {\tt whisky::rho}
      then the first entry of this array is found from {\tt
        CCTK\_VarIndex("whisky::rho")}.
    \item {\tt Dependent GFs}: An array of integers containing the
      indices of the dependent grid functions, following the syntax of
      the {\tt Independent GFs}.
    \end{itemize}
  \end{Discussion}

  \begin{SeeAlsoSection}
    \begin{SeeAlso}{EOS\_SetGFs}
      Sets the EOS.
    \end{SeeAlso}
  \end{SeeAlsoSection}
\end{FunctionDescription}



\begin{FunctionDescription}{EOS\_SetGFs}
  \label{Whisky_Dev_EOSBase_General_EOS_SetGFs}  
  
  Actually sets the dependent GFs in an EOS call. The call must have
  been previously set up using EOS\_SetupCall. Must be called from
  routines scheduled in {\tt local} mode.

  \begin{SynopsisSection}
    \begin{Synopsis}{C}
\begin{verbatim}
CCTK_INT ierr = EOS_SetGFs(CCTK_POINTER_TO_CONST cGH, CCTK_INT call_handle)
\end{verbatim}
    \end{Synopsis}
    \begin{Synopsis}{Fortran}
\begin{verbatim}
CCTK_INT ierr = EOS_SetGFs(CCTK_POINTER_TO_CONST cGH, CCTK_INT call_handle)
\end{verbatim}
    \end{Synopsis}
  \end{SynopsisSection}

  \begin{ResultSection}
    \begin{ResultNote}
      The {\tt call\_handle} should have been previously set up by
      EOS\_SetupCall. Currently only successful returns (0) are
      given.
    \end{ResultNote}
    \begin{Result}{\rm 0}
      success.
    \end{Result}
  \end{ResultSection}

  \begin{ParameterSection}
    \begin{Parameter}{cGH}
      The Cactus GH structure, and the reason why this must be called
      in {\tt local} mode.
    \end{Parameter}
    \begin{Parameter}{call\_handle}
      An EOS call previously set up by EOS\_SetupCall.
    \end{Parameter}
  \end{ParameterSection}

  \begin{Discussion}
    Once the EOS call has been setup this routine basically tells the
    EOS function to apply the appropriate EOS calls at this point to
    the appropriate arrays.
  \end{Discussion}

  \begin{SeeAlsoSection}
    \begin{SeeAlso}{EOS\_SetupCall}
      Setup the EOS call.
    \end{SeeAlso}
  \end{SeeAlsoSection}
\end{FunctionDescription}

\pagebreak
\section{Registering an EOS}

Before you can use an EOS one first has to be registered. The best way
of learning how to do this is to look at the example thorns, such as
the ``generalized'' polytropic EOS (thorn {\tt EOS\_GeneralPolytrope}) or
ideal gas EOS (thorn {\tt EOS\_GeneralIdealFluid}). Here is just the
essential description.

The essential thing that is required is a function (probably in C,
although in theory it's language independent) that sets arbitrary
sized arrays\footnote{The arbitrariness applies to the different
  calls, not different arrays: all arrays, whether for independent or
  dependent variables are assumed to have the same size in any given
  call.} of dependent variables from the independent variables. This
function is registered using the aliased function {\tt
  EOS\_RegisterCall}. This function, and the function pointer that
must be called, are as follows.




\begin{FunctionDescription}{EOS\_RegisterCall}
  \label{Whisky_Dev_EOSBase_General_EOS_RegisterCall}  
  
  Register an EOS with the base thorn.

  \begin{SynopsisSection}
    \begin{Synopsis}{C}
\begin{verbatim}
CCTK_INT ierr = EOS_RegisterCall(CCTK_INT table_handle, \
                                 CCTK_INT CCTK_FPOINTER EOS_fn(...))
\end{verbatim}
    \end{Synopsis}
    \begin{Synopsis}{Fortran}
\begin{verbatim}
CCTK_INT ierr = EOS_RegisterCall(CCTK_INT table_handle, \
                                 CCTK_INT CCTK_FPOINTER EOS_fn(...))
\end{verbatim}
    \end{Synopsis}
  \end{SynopsisSection}

  \begin{ResultSection}
    \begin{ResultNote}
      The function pointer {\tt EOS\_fn} is described below on
      page~\pageref{Whiskty_Dev_EOSBase_General_EOS_fn}.  

      Currently only successful returns (0) are given.
    \end{ResultNote}
    \begin{Result}{\rm 0}
      success.
    \end{Result}
  \end{ResultSection}

  \begin{ParameterSection}
    \begin{Parameter}{table\_handle}
      A key/value table containing any persistent information that the
      EOS thorn may want to store. This is not used by the base thorn,
      and is there for the convenience of the EOS thorn.
    \end{Parameter}
    \begin{Parameter}{EOS\_fn}
      A function pointer for the EOS call itself, described below.
    \end{Parameter}
  \end{ParameterSection}

  \begin{Discussion}
    The table can be used to differentiate between different
    calls. For example, if one thorn implements many different
    EOSs and the number is not known at compile time (e.g., a table
    interpolator thorn that provides multiple tables) then only one
    registration need take place, and the key/value table can be used
    store which (interpolation) table is being used in the particular
    call. 
  \end{Discussion}

  \begin{SeeAlsoSection}
    \begin{SeeAlso}{EOS\_fn}
      The actual EOS function.
    \end{SeeAlso}
  \end{SeeAlsoSection}
\end{FunctionDescription}


\begin{FunctionDescription}{EOS\_fn}
  \label{Whisky_Dev_EOSBase_General_EOS_fn}  
  
  The actual EOS function that must be provided.

  \begin{SynopsisSection}
    \begin{Synopsis}{C}
\begin{verbatim}
CCTK_INT ierr = EOS_fn(CCTK_INT param_table, \
                       CCTK_INT n_elems, \
                       CCTK_POINTER* indep_arrays, \
                       CCTK_INT* which_dep_arrays_to_set, \
                       CCTK_POINTER* dep_arrys)
\end{verbatim}
    \end{Synopsis}
    \begin{Synopsis}{Fortran}
\begin{verbatim}
CCTK_INT ierr = EOS_fn(CCTK_INT param_table, \
                       CCTK_INT n_elems, \
                       CCTK_POINTER* indep_arrays, \
                       CCTK_INT* which_dep_arrays_to_set, \
                       CCTK_POINTER* dep_arrys)
\end{verbatim}
    \end{Synopsis}
  \end{SynopsisSection}

  \begin{ResultSection}
    \begin{ResultNote}
      Currently only successful returns (0) are given.
    \end{ResultNote}
    \begin{Result}{\rm 0}
      success.
    \end{Result}
  \end{ResultSection}

  \begin{ParameterSection}
    \begin{Parameter}{param\_table}
      A key/value table containing any persistent information that the
      EOS thorn may want to store. This is not used by the base thorn,
      and is there for the convenience of the EOS thorn.
    \end{Parameter}
    \begin{Parameter}{n\_elems}
      The size of the arrays that are passed in.
    \end{Parameter}
    \begin{Parameter}{indep\_arrays}
      The independent arrays (should essentially be cast to a {\tt
        CCTK\_REAL**}).
    \end{Parameter}
    \begin{Parameter}{which\_dep\_arrays\_to\_set}
      An array telling the EOS thorn which dependent arrays should be
      set in this thorn. This is important because dependent arrays
      that are not needed will NOT be passed in the call; see below.
    \end{Parameter}
    \begin{Parameter}{dep\_arrays}
      The dependent arrays (should essentially be cast to a {\tt
        CCTK\_REAL**}). Note that you should not assume that all these
      pointers are non-null (hence the difficulty in using this with
      Fortran). In general, if the calling thorn does NOT request that
      a dependent variable be set then the equivalent pointer WILL be
      null.
    \end{Parameter}
  \end{ParameterSection}

  \begin{Discussion}
    The table can be used to differentiate between different
    calls. For example, if one thorn implements many different
    EOSs and the number is not known at compile time (e.g., a table
    interpolator thorn that provides multiple tables) then only one
    registration need take place, and the key/value table can be used
    store which (interpolation) table is being used in the particular
    call. 
  \end{Discussion}

  \begin{SeeAlsoSection}
    \begin{SeeAlso}{EOS\_RegisterCall}
      Register the EOS function with the base thorn.
    \end{SeeAlso}
  \end{SeeAlsoSection}
\end{FunctionDescription}



\begin{thebibliography}{9}

\end{thebibliography}

% Do not delete next line
% END CACTUS THORNGUIDE

\end{document}
