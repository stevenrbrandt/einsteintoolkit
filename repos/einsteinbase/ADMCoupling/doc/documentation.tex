\documentclass{article}

% Use the Cactus ThornGuide style file
% (Automatically used from Cactus distribution, if you have a 
%  thorn without the Cactus Flesh download this from the Cactus
%  homepage at www.cactuscode.org)
\usepackage{../../../../doc/latex/cactus}

\begin{document}

\title{ADMCoupling}
\author{Ian Hawke, David Rideout}
\date{$ $Date$ $}

\maketitle

% Do not delete next line
% START CACTUS THORNGUIDE

\begin{abstract}
This thorn allows seamless coupling of evolution and analysis thorns
to any thorns which contribute matter terms to the stress energy
tensor $T_{ab}$.
\end{abstract} 

\section{Purpose}

This thorn is completely trivial (there's already more words in the
documentation than in the code). The point is to allow clean coupling
of matter thorns and spacetime evolution thorns. By making a spacetime
thorn (such as BSSN) a friend of \texttt{ADMCoupling} it can know about the
variables of the matter thorns (such as Whisky) as long as they are
also friends of \texttt{ADMCoupling}, and then the appropriate stress energy
tensor terms can be included through the {\tt CalcTmunu} interface.
This avoids explicit dependencies between the spacetime and matter
evolution thorns. Note that we need to do the same for certain
analysis thorns, such as \texttt{ADMConstraints}.

\section{CalcTmunu}

\subsection{Background}
So what is this ``\texttt{CalcTmunu}'', anyway?
\texttt{CalcTmunu} is a general interface which allows any thorn to
`declare' that it contains matter variables, by adding terms to the
components of the stress energy tensor.  This is done using the Cactus
include file mechanism, which allows thorns to contribute code to
include files, which can then be included by any thorn which wishes to
use them.
For \texttt{CalcTmunu} there are two include files\footnote{There used
to be a third, \texttt{CalcTmunu\_rfr.inc}, which was a Cactus 3 legacy
having to do with the rfr scheduling mechanism.  You may safely delete
any reference to this file.},
\texttt{CalcTmunu.inc} and \texttt{CalcTmunu\_temps.inc}.  
%The name ``temps'' is not so great, it would be better to have the
%one ``source'' and the other ``declarations'', or something.

In \texttt{CalcTmunu.inc}, one can place code of the form
\begin{verbatim}
	Ttt = Ttt + ...
	Ttx = Ttx + ...
	Tty = Tty + ...
	Ttz = Ttz + ...
	Txx = Txx + ...
	Txy = Txy + ...
	Txz = Txz + ...
	Tyy = Tyy + ...
	Tyz = Tyz + ...
	Tzz = Tzz + ...
\end{verbatim}
to add terms to the components $T_{\mu\nu}$.  Each of these variables
is of type \texttt{CCTK\_REAL}.  (If you omit the
\texttt{Ttt \nolinebreak+}, \texttt{Ttx \nolinebreak+}, etc. from the right hand sides of
these assignment statements then you will be assuming that your thorn
is the only one which provides matter degrees of freedom, and
excluding contributions from other matter thorns which may be
activated.)
This code will be
executed for each point on the grid, whose indices will be stored in
the integers \texttt{i}, \texttt{j}, and \texttt{k}.  Currently it must be 
`fixed form' Fortran code.

\texttt{CalcTmunu\_temps.inc} will be
included in the variable declaration section for the block of code
which contains the
\nopagebreak
\begin{verbatim}
#include "CalcTmunu.inc"
\end{verbatim}
One can put local temporary variable declarations needed for the code
above into this file.  The \texttt{Ttt}, \texttt{Ttx}, etc.~will be
declared within a macro from \texttt{ADMMacros}.

\subsection{For matter thorns}
To make use of the {\tt CalcTmunu} interface, simply place the lines
\begin{verbatim}
INCLUDES HEADER: <MyThorn_CalcTmunu_temps.inc> in CalcTmunu_temps.inc
INCLUDES SOURCE: <MyThorn_CalcTmunu.inc> in CalcTmunu.inc
\end{verbatim}
in your thorn's \texttt{interface.ccl} file, and declare your thorn to
be friends with \texttt{ADMCoupling}.  Then provide the files
\texttt{<MyThorn\_CalcTmunu\_temps.inc>} and
\texttt{<MyThorn\_CalcTmunu.inc>} somewhere in your thorn's source
code.
We expect to add a sample matter thorn to \texttt{CactusEinstein}
soon, which will illustrate the use of this.

\subsection{For thorns which need the stress energy tensor}
Spacetime evolution thorns and various analysis thorns (e.g. one which
computes the constraints) may need to
know the value of the stress-energy tensor.
See thorn \texttt{CactusEinstein/ADM} for an example the former.
%of a thorn which
%uses the code inserted into the \texttt{CalcTmunu} include files.
There the macro \texttt{KSOURCES} (see \texttt{KSOURCES\_declare.h}
and \texttt{KSOURCES\_guts.h}) is called from the various time
integration source files \texttt{DoubleLeap.F},
\texttt{IterativeCN.F}, etc., which in turn include the
\texttt{CalcTmunu\_temps.inc} and \texttt{CalcTmunu.inc} files.  In
this example, the
$T_{\mu\nu}$ temporary variables are provided by including
\texttt{CactusEinstein/ADMMacros/src/macro/TRT\_declare.h}.

% Do not delete next line
% END CACTUS THORNGUIDE

\end{document}
% LocalWords:  ADMCoupling Hawke Whisky CalcTmunu ADMConstraints Ttt Ttx ccl
% LocalWords:  ADMMacros MyThorn KSOURCES DoubleLeap IterativeCN rfr CCTK
