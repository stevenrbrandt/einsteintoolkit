% /*@@
%   @file      Preface.tex
%   @date      27 Jan 1999
%   @author    Tom Goodale, Gabrielle Allen, Gerd Lanferman
%   @desc 
%   Preface for the Cactus User's Guide
%   @enddesc 
%   @version $Header$      
%   @history
%   @date       Sun Jul 20 12:57:28 CEST 2003
%   @author     Jonathan Thornburg
%   @desc       split off reference material into a separate
%               Reference Manual (in new directory ../ReferenceManual/)
%   @endhistory
% @@*/

%%%%%%%%%%%%%%%%%%%%%%%%%%%%%%%%%%%%%%%%%%%%%%%%%%%%%%%%%%%%%%%%%%%%%%%%%%%%%%%%
%%%%%%%%%%%%%%%%%%%%%%%%%%%%%%%%%%%%%%%%%%%%%%%%%%%%%%%%%%%%%%%%%%%%%%%%%%%%%%%%

{\large \bf Preface} 
\label{sec:pr}
 
\vskip .5cm

This document contains a quick-start guide to installing and running
a Cactus application. In subsequent chapters, it provides more detailed
information on advanced user's topics, as well as an introduction to 
thorn writing.
Please report omissions, errors, or suggestions to 
any of our contact addresses below. 

\vskip .5cm

\textbf{Overview of documentation}

\begin{Lentry}

\item [\textbf{Part~\ref{part:Introduction}: Introduction to Cactus.}]
  A guide through the process of obtaining and installing Cactus and running
  a simple example application with it. 

\item [\textbf{Part~\ref{part:Notes}: Additional Notes.}]
  A more in-depth description of required hardware and software, along with 
  configuration, installation and running options. 
  Describes how to check the installation with Cactus test suites. 

\item [\textbf{Part~\ref{part:ThornWriting}: Thorn Writing.}]
  An introduction to thorn concepts and description of how to create, write
  and maintain application thorns. Explanation of use of the
  programming interface to take advantage of parallelism and modularity.
  This is followed by a more advanced discussion of user supplied 
  infrastructure routines such
  as additional \textit{output routines}, \textit{drivers}, etc.

\item [\textbf{Part~\ref{part:Appendices}: Appendices.}]
        These contain a glossary, a description of the Cactus Configuration 
	Language, the Utility routines
        and other odds and ends, such as how to use GNATS and TAGS.

\end{Lentry}

Related topics are discussed in separate documents including:

\begin{Lentry}

\item [\textbf{Reference Manual}] Contains detailed descriptions of
  the functions provided by the Cactus flesh API, along with
  other reference material.

%\item [{\bf Computational Thorn Guide}] This will contain details about the 
%arrangements and thorns making up the standard Cactus Computational Tool Kit.

%\item [{\bf Relativity Thorn Guide}] This will contain details about the arrangements and thorns making up the Cactus Relativity Tool Kit, one of the major 
% motivators, and still the driving force, for the Cactus Code.

%\item [{\bf Flesh Maintainers Guide}] 
% This will contain all the gruesome details
% about the inner workings of Cactus, for all those who want or need to 
% expand or maintain the core of Cactus.

\end{Lentry}

\vskip .5cm

{\bf Typographical Conventions}

\begin{Lentry}

\item[{\tt Typewriter}] Is currently used for everything you type,
	for program names, and code extracts.
\item[{\tt < ... >}] Indicates a compulsory argument.
\item[{\tt [ ... ]}] Indicates an optional argument.
\item[{\tt |}] Indicates an exclusive or.

\end{Lentry}
 
\vskip .5cm

{\bf How to Contact Us}

\vskip .5cm

Please let us know of any errors or omissions in this guide, as well
as suggestions for future editions. These can be reported 
via email to
\code{cactusmaint@cactuscode.org}.

\vskip .5cm

{\bf Acknowledgements}

\vskip .5cm

Hearty thanks to all those who have helped with documentation for the
Cactus Code. Special thanks to those who struggled with the earliest
sparse versions of this guide and sent in mistakes and suggestions,
in particular John Baker, Carsten Gundlach, Ginny Hudak-David, 
Sai Iyer, Paul Lamping, Nancy Tran and Ed Seidel. 

%%%%%%%%%%%%%%%%%%%%%%%%%%%%%%%%%%%%%%%%%%%%%%%%%%%%%%%%%%%%%%%%%%%%%%%%%%%%%%%%
%%%%%%%%%%%%%%%%%%%%%%%%%%%%%%%%%%%%%%%%%%%%%%%%%%%%%%%%%%%%%%%%%%%%%%%%%%%%%%%%
