% $Header$

\documentclass[12pt]{article}

\begin{document}

\title{Solving Nonlinear Elliptic Equations:\\ Calling PETSc from
  Cactus}

\author{Erik Schnetter $<$schnetter@uni-tuebingen.de$>$}

\date{August 28, 2001}

\maketitle

\begin{abstract}
  The Cactus thorn \texttt{TATPETSc} provides a simple interface to
  the SNES interface of PETSc.  SNES (``Simple Nonlinear Elliptic
  Solver'') is a set of efficient parallel solvers for sparse
  matrices, i.e.\ for discretised problems with small stencils.  The
  main task of \texttt{TATPETSc} is in handling the mismatch in the
  different parallelisation models.
\end{abstract}

\tableofcontents

\section{Introduction}

PETSc (``The Portable, Extensible Toolkit for Scientific
Computation'') \cite{TAT:TATPETSc:petsc} is a library that, among
other things, solves nonlinear elliptic equations.  It is highly
configurable, and using it is a nontrivial exercise.  It is therefore
convenient to create wrappers for this library that handle the more
common cases.  Although this introduces a layer that is a black box to
the PETSc novice, it at the same time enables this PETSc novice to use
this library, which would otherwise not be possible.

At the moment there exist three wrapper routines.  Two of them handle
initialisation and shutting down the library at programme startup and
shutdown time.  The third solves a nonlinear elliptic equation.

\section{Nonlinear Elliptic Equations}

Nonlinear elliptic equations can be written in the form
$$
F(x) = 0
$$
where $x$ is an $n$-dimensional vector, and $F$ is function with an
$n$-dimensional vector as result.  The vector $x$ is the unknown to
solve for.  The function $F$ has to be given together with initial
data $x_0$ for the unknown $x$.

In the case of discretised field equations, the vector $x$ contains
all grid points.  The function $F$ then includes the boundary
conditions.

Consider as an example the Laplace equation $\Delta \phi = 0$ in three
dimensions on a grid with $10^3$ points.  In that case, the unknown
$x$ is $\phi$, with the dimension $n=1000$.  The function $F$ is given
by $F(x) = \Delta x$.  Remember that, once a problem is discretised,
derivative operators are algebraic expressions.  That means that
calculating $\Delta x$ does not involve taking analytic (``real'')
derivatives, but can rather be written as linear function (matrix
multiplication).

\section{Solving a Nonlinear Elliptic Equation}

The wrapper routine that solves a nonlinear elliptic equation is

\texttt{
\begin{tabbing}
void \=TATPETSc\_solve (\\
\>const cGH *cctkGH,\\
\>const int *var,\\
\>const int *val,\\
\>int nvars,\\
\>int options\_table,\\
\>void (*fun) (cGH *cctkGH, int options\_table, void *userdata),\\
\>void (*bnd) (cGH *cctkGH, int options\_table, void *userdata),\\
\>void *data);
\end{tabbing}
}

There is currently no Fortran wrapper for this routine.

This routine takes the following arguments:
\begin{description}
   \item[\texttt{cctkGH}] a pointer to the Cactus grid hierarchy

   \item[\texttt{var}] the list of indices of the grid variables $x$
   to solve for

   \item[\texttt{val}] the list of indices of the grid variables that
   contain the function value $F(x)$

   \item[\texttt{nvars}] the number of variables to solve for, which
   is the same as the number of function values

   \item[\texttt{options\_table}] a table with additional options (see
   below)

   \item[\texttt{fun}] the routine that calculates the function values
   from the variables

   \item[\texttt{bnd}] the routine that applies the boundary
   conditions to the variables

   \item[\texttt{data}] data that are passed through unchanged to the
   callback routines
\end{description}

The options table can have the following elements:
\begin{description}
   \item[\texttt{CCTK\_INT periodic[dim]}] $dim$ flags indicating
   whether the grid is periodic in the corresponding direction.
   ($dim$ is the number of dimensions of the grid variables.)  The
   default values are 0.

   \item[\texttt{CCTK\_INT solvebnds[2*dim]}] $2\cdot dim$ flags
   indicating whether the grid points on the corresponding outer
   boundaries should be solved for (usually not).  The default values
   are 0.

   \item[\texttt{CCTK\_INT stencil\_width}] the maximum stencil size
   used while calculating the function values from the variables
   (should be as small as possible).  The default value is 1.

   \item[\texttt{CCTK\_FN\_POINTER jacobian}] The real type of
   \texttt{jacobian} must be \texttt{void (*jacobian)(cGH *cctkGH,
   void *data)}.  This is either a routine that calculates the
   Jacobian directly, or 0 (zero).  The default is 0.

   \item[\texttt{CCTK\_FN\_POINTER get\_coloring}] The real type of
   \texttt{get\_coloring} must be \texttt{void (*get\_coloring)(DA da,
   ISColoring *iscoloring, void *userdata)}.  This is either a routine
   that calculates the colouring for calculating the Jacobian, or 0
   (zero).  You need to pass a nonzero value only if you have very
   special boundary conditions.  The default is 0.
\end{description}

A few remarks:
\begin{itemize}

   \item In order to be able to call this routine directly, it is
   necessary to inherit from TATPETSc in the thorn where this routine
   is called.
  
   \item Increasing \texttt{stencil\_width} from 1 to 2 increases the
   run time by about a factor of 5.  In general you want to be able to
   set \texttt{stencil\_width} to 1.  Note that calculating $F$ does
   normally not require upwind derivatives with their larger size.
   \texttt{stencil\_width} does not have to be equal to the number of
   ghost zones.
  
   \item One sure way to speed up solving nonlinear elliptic equations
   is to provide an explicit function that calculates the Jacobian.
   While such a function is in principle straightforward to write, it
   is very (very!) tedious to do so.  If you pass 0 for this function,
   the Jacobian is evaluated numerically.  This is about one order of
   magnitude slower, but is also a lot less work.
\end{itemize}

\section{Initial data}

A linear elliptic equation does not need initial data.  However, a
nonlinear elliptic equation does.  It may have several solutions, and
the initial data select between these solutions.  The initial data
$x_0$ have to be put into the variables $x$ before the solver is
called.  The function values $F(x)$ can remain undefined.

\section{Boundary Conditions}

There are two functions that you have to apply boundary conditions to.
You need to apply boundary conditions to the right hand side, i.e.\
$F$, and to the solution, i.e.\ $x$.

The routine \texttt{bnd} has to apply boundary conditions to $x$ at
exactly those boundaries where \texttt{solvebnds} is false.  (The
boundaries where \texttt{solvebnds} is true have already been
determined by the solver.)  This includes imposing the symmetry
boundary conditions.  Not applying a boundary condition on the outer
boundaries is equivalent to a Dirichlet boundary condition, with the
boundary values given in the initial data $x_0$.

After calculating $F$, you need to apply the necessary boundary
conditions to $F$ as well.  You have to do this in the routine
\texttt{fun} that calculates $F$.

\section{The Solution, Or Not}

When the solver returns, the result is available in $x$.  $F(x)$
contains the corresponding function value, which should be close to
zero.

It is possible that the solver doesn't converge.  In this case, $F$
will not be close to zero.

It is possible (by the way of programming error) to make certain grid
points in $F$ independent of $x$, or to ignore certain grid points in
$x$ while calculating $F$.  This leads to a singular matrix when the
nonlinear solver calls a linear solver as a substep.  Such a system is
ill-posed and cannot be solved.

\section{Excision}

An excision boundary leads to a certain number of grid points that
should not be solved for.  In order to avoid a singular matrix, it is
still necessary to impose a condition on these grid points.  Assuming
that you want a Dirichlet-like condition for these grid points, I
suggest
$$
F(x) = x - x_0
$$
where $x_0$ are the initial data that you have to save someplace.
Note that you have to impose this (or a different) condition not only
onto the boundary points, but onto all excised points, i.e.\ all
points that are not solved for.

(The above condition satisfies $F(x)=0$ for $x=x_0$, which will be the
solution for these grid points.  Setting $F(x)=0$ does \emph{not} work
because it leads to a singular matrix, as outlined above.)

\section{Common PETSc options}

Options Database Keys (from the PETSc documentation)

\begin{description}

\item[\texttt{-snes\_type} \textit{type}]
	ls, tr, umls, umtr, test 

\item[\texttt{-snes\_stol}]
	convergence tolerance in terms of the norm of the change in
	the solution between steps

\item[\texttt{-snes\_atol} \textit{atol}]
	absolute tolerance of residual norm

\item[\texttt{-snes\_rtol} \textit{rtol}]
	relative decrease in tolerance norm from initial

\item[\texttt{-snes\_max\_it} \textit{max\_it}] 
	maximum number of iterations

\item[\texttt{-snes\_max\_funcs} \textit{max\_funcs}]
	maximum number of function evaluations 

\item[\texttt{-snes\_trtol} \textit{trtol}]
	trust region tolerance 

\item[\texttt{-snes\_no\_convergence\_test}]
	skip convergence test in nonlinear or minimization solver;
	hence iterations will continue until \textit{max\_it} or some
	other criterion is reached.  Saves expense of convergence test

\item[\texttt{-snes\_monitor}] 
	prints residual norm at each iteration 

\item[\texttt{-snes\_vecmonitor}]
	plots solution at each iteration 

\item[\texttt{-snes\_vecmonitor\_update}]
	plots update to solution at each iteration 

\item[\texttt{-snes\_xmonitor}]
	plots residual norm at each iteration 

\item[\texttt{-snes\_fd}]
	use finite differences to compute Jacobian; very slow, only
	for testing

\item[\texttt{-snes\_mf\_ksp\_monitor}]
	if using matrix-free multiply then print $h$ at each KSP
	iteration

\end{description}

\section{Installing PETSc}

Before you can use TATPETSc, you have to install PETSc.  PETSc comes
with extensive documentation for that.

In order to be able to use PETSc with Cactus, you have to give certain
options when you configure your Cactus applications.  To do so, create
an options file containing the following options.

\begin{description}
   \item{SYS\_INC\_DIRS} /usr/include/petsc

   \item{LIBDIRS} /usr/X11R6/lib

   \item{LIBS} crypt petscfortran petscts petscsnes petscsles petscdm
   petscmat petscvec petsc lapack blas mpe mpich X11 g2c z
\end{description}

Replace /usr/include/petsc with the corresponding directory on your
machine.

If you have other packages that need other options, then you have to
combine these options manually.  Even if the other options would
normally be selected automatically, selecting the PETSc options
manually will override the other options.

Do not forget to activate MPI.  PETSc needs MPI to run.

\begin{thebibliography}{99}
   \bibitem{TAT:TATPETSc:petsc} PETSc:
   \verb+http://www-fp.mcs.anl.gov/petsc/+
\end{thebibliography}

\end{document}
