\documentclass{article}

% Use the Cactus ThornGuide style file
% (Automatically used from Cactus distribution, if you have a 
%  thorn without the Cactus Flesh download this from the Cactus
%  homepage at www.cactuscode.org)
\usepackage{../../../../doc/latex/cactus}

\begin{document}

\title{EllSOR}
\author{Joan Masso}
\date{$ $Date$ $}

\maketitle

% Do not delete next line
% START CACTUS THORNGUIDE

\begin{abstract}
{\tt EllSOR} provides 3D elliptic solvers for the various
classes of elliptic problems defined in {\tt EllBase}. {\tt EllSOR} is 
based on the successive over relaxation algorithm. It is called by the
interfaces provided in {\tt EllBase}.
\end{abstract}

\section{Purpose}
The purpose of this thorn is to provide a simple and straightforward
3D elliptic solver: not to be used by production but to demonstrate key
features of the elliptic infrastructure.

This thorn provides
 \begin{enumerate}
  \item No Pizza
  \item No Wine
  \item peace
 \end{enumerate}

\section{Technical Details}
This thorn supports three elliptic problem classes: {\bf LinFlat} for 
a standard 3D cartesian Laplace operator, using the standard 7-point
computational molecule. {\bf LinMetric} for a Laplace operator derived
from the metric, using 19-point stencil. {\bf LinConfMetric} for a
Laplace operator derived from the metric and a conformal factor, using 
a 19-point stencil. The code of the solvers differs for the classes
and is explained in the following section. 

In general, a stencil variable needs to be set for each of the
direction relative to the central gridpoint. These variables are
called {\tt ac}, {\tt ae}, {\tt aw}, {\tt an}, {\tt as}, {\tt at}, {\tt ab}, {\tt
ane}, {\tt anw}, {\tt ase}, {\tt asw}, {\tt ate}, {\tt atw}, {\tt abe}, {\tt
abw}, {\tt atn}, {\tt ats}, {\tt abn}, {\tt asb}, where ``{\tt ac}'' =
a-central, ``{\tt t}'' = top, ``{\tt b}'' = bottom, ``{\tt n,s,w,e}'' = north, south, west, east

\subsection{{\bf LinFlat}}
For this class we employ the the 7-point stencil based on {\tt at,ab,
aw, ae, an, as} only. These values are constant at each gridpoint.

\subsection{{\bf LinMetric}}
For this class the standard 19-point stencil is initialized, taken the 
underlying metric into account. The values for the stencil function
differ at each gridpoints.

\subsection{{\bf LinConfMetric}}
For this class the standard 19-point stencil is initialized, taken the 
underlying metric and its conformal factor into account. The values
for the stencil function differ at each gridpoints.

\section{Comments}
The sizes of the arrays {\tt Mlinear} for the coefficient matrix and
{\tt Nsource} are passed in the solver. A storage flag is set if these 
variables are of a sized greater 1. In this case, the array can be
accessed.

%\section{My own section}

% Do not delete next line
% END CACTUS THORNGUIDE

\end{document}
